\documentclass[a4paper, 12pt]{article}
\usepackage[usenames, dvipsnames]{color}%% textcolor
\usepackage{listings}%% Listing code
\usepackage{color}	%color code
%	define the code style
\definecolor{codegreen}{rgb}{0,0.6,0}
\definecolor{codegray}{rgb}{0.5,0.5,0.5}
\definecolor{codepurple}{rgb}{0.58,0,0.82}
\definecolor{backcolour}{rgb}{0.95,0.95,0.92}
 
\lstdefinestyle{mystyle}{
    backgroundcolor=\color{backcolour},   
    commentstyle=\color{codegreen},
    keywordstyle=\color{magenta},
    numberstyle=\tiny\color{codegray},
    stringstyle=\color{codepurple},
    basicstyle=\footnotesize,
    breakatwhitespace=false,         
    breaklines=true,                 
    captionpos=b,                    
    keepspaces=true,                 
    numbers=left,                    
    numbersep=5pt,                  
    showspaces=false,                
    showstringspaces=false,
    showtabs=false,                  
    tabsize=2
}
 
\lstset{style=mystyle}
%% ref: https://www.sharelatex.com/learn/Code_listing 
\title{\LaTeX Tutorial for the uninitiated}
\author{Tongust}
\renewcommand*\contentsname{Summary}
\begin{document}
    \maketitle
	\tableofcontents
%%%%The verbatim environment
    \section{The verbatim environment}

	\emph{The default tool to display code in LATEX is verbatim, which generates an output in monospaced font.}

	Text enclosed inside \texttt{verbatim} environment 
	is printed directly 
	and all \LaTeX{} commands are ignored.	
   \begin{verbatim}
		Text enclosed inside \texttt{verbatim} environment 
		is printed directly 
		and all \LaTeX{} commands are ignored.
	\end{verbatim} 
%%%%%The listings environment
    \section{Code Listing}

	This is code listing.

	\begin{lstlisting}[language=Python, caption=Python example]
		import numpy as np
		def my_fun():
			print "Hello LaTeX"
			vt = np.zeros(1, int)	# dummy variable
		# end of function
			return 0
	\end{lstlisting}
%%%%%% Using colours in LaTeX
    \section{Using colours in \LaTeX}

	This example shows different examples on how to use the \texttt{color} package 
	to change the colour of elements in \LaTeX.
 
	\begin{itemize}
	\color{ForestGreen}
	\item First item
	\item Second item
	\end{itemize}
	 
	\noindent
	{\color{RubineRed} \rule{\linewidth}{0.5mm} }
	 
	The background colour of some text can also be \textcolor{red}{easily} set. For 
	instance, you can change to orange the background of \colorbox{BurntOrange}{this 
	text} and then continue typing.



    %%%%%%%%%%%%%%%%%%%%%%%% End of doc

	\addcontentsline{toc}{section}{List of Listings}
	\section*{List of Listings}

	\emph{Adding the comma-separated parameter caption=Python example inside the brackets, enables the caption. This caption can be later used in the list of Listings.}

	\lstlistoflistings

\end{document}
