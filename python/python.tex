\documentclass[a4paper, 12pt]{article}
\usepackage[usenames, dvipsnames]{color}%% textcolor
\usepackage{listings}%% Listing code
\usepackage{color}	%color code
%	define the code style
\definecolor{codegreen}{rgb}{0,0.6,0}
\definecolor{codegray}{rgb}{0.5,0.5,0.5}
\definecolor{codepurple}{rgb}{0.58,0,0.82}
\definecolor{backcolour}{rgb}{0.95,0.95,0.92}
 
\lstdefinestyle{mystyle}{
    backgroundcolor=\color{backcolour},   
    commentstyle=\color{codegreen},
    keywordstyle=\color{magenta},
    numberstyle=\tiny\color{codegray},
    stringstyle=\color{codepurple},
    basicstyle=\footnotesize,
    breakatwhitespace=false,         
    breaklines=true,                 
    captionpos=b,                    
    keepspaces=true,                 
    numbers=left,                    
    numbersep=5pt,                  
    showspaces=false,                
    showstringspaces=false,
    showtabs=false,                  
    tabsize=2
}
\lstset{style=mystyle}
%%%%%%%%%%%%%%%%%%%%%%%%%%%%%%%%%%%%%%%%%%%%%%%%%%%%%%%%%%%
%%%%%%%%%%%%%%%%%%% END OF HEAD FILE %%%%%%%%%%%%%%%%%%%%%%
%%%%%%%%%%%%%%%%%%%%%%%%%%%%%%%%%%%%%%%%%%%%%%%%%%%%%%%%%%%
\title{Python Note}

\author{tonguste@gmail.com}
\renewcommand*\contentsname{Summary}
\begin{document}
    \maketitle
	\tableofcontents
    %%%%%%%%%%%%%%%%%% SECTION  %%%%%%%%%%%%%%%%%%%%%%%%%%%
    \section{python max function using key and lambda expression}
	By default in Python 2 key compares items based on a set of rules based on the type of the objects(for example a string is always greater than an integer).
	To modify the object before comparison or to compare based on a particular attribute/index you've to use the key argument.\\
	\textbf{Example 1:}
	
	\begin{lstlisting}	[language=Python, caption=lambda and max key]
		>>>lis = [1,2,3,'4','5','14']
		>>>max(lis)
		WRONG!!!
		>>>max(lis, key=lambda x: int(x))
		# change the value to be order into interge
		>>>lis = [(1,'w'),(2,'r'),(4,'g')]
		>>>max(lis, key = lambda x:x[1])

	\end{lstlisting}

    \section{List indices must be integers, not list}
    List in Python must be integers, such as 
    \begin{lstlisting} [language=Python]
        a = [1,2,3,4]
        a[1]    #will return 2
        a[[1,0]] # it wont echo 2 1 as expected.
        # via numpy
        import numpy as np
        a = np.array(a)
        a[[1,0]] # Will return 2,1
    \end{lstlisting}
    However, array in \textbf{numpy} can be indiced by a list.

    \section{numpy exchange between 1d array and 2d array}
    Sclicing the $i^{th}$ column in numpy is totally different with MatLab.
    \begin{lstlisting} [language=Python]
        import numpy as np
        c = np.arange(16).reshape((4,4))
		#  [[ 0,  1,  2,  3],
	    #   [ 4,  5,  6,  7],
    	#   [ 8,  9, 10, 11],
      	#   [12, 13, 14, 15]]
		# if you want extract the 1 column, namely, [[1],[5],[9], [13]], you cannot use c[:, 1] or c[:][1] which doesn't return a vector
		c[:, 1]
		# array([1,  5,  9, 13])
		c[:]
		# array([4, 5, 6, 7]), surprisely!
		# If you want change the some some column of c, you can do as:
		c[:,1] = np.zeros((4,1))
		# Oops...
		# Traceback (most recent call last):
	  	# File "<stdin>", line 1, in <module>
		# TypeError: data type not understood
		
		c[:, 1] = np.zeros((4,1)).shape((1,4))
		# Wrong again!!
		np.zeros((4,1)).shape((4,1))
		# array([[ 0.,  0.,  0.,  0.]]) A matrix NOT vector
		# right way:
		c[:, 1] = np.zeros((4,1)).shape(4)
		# Okay, np.array.shape() (without tuple) will return a vector
		#array([[ 0,  0,  2,  3],
		#       [ 4,  0,  6,  7],
		#       [ 8,  0, 10, 11],
		#       [12,  0, 14, 15]])
    \end{lstlisting}
    %%%%%%% SECTION %%%%%%%
    \section{IPython with Pyspark}
    We can use IPYTHON when lanuching the Pyspark console.\\
    \begin{lstlisting} [language=Perl]
    $ipython=1 IPYTHON_OPTS="--pylab" $SPARK_HOME/bin/pyspark
    \end{lstlisting}


    %%%%%%%%%%%%%%%%% end of section %%%%%%%%%%%%%%%%%%%%%
    
    


    %%%%%%%%%%%%%%%%%% End of doc %%%%%%%%%%%%%%%%%%%%%%%%%

	\addcontentsline{toc}{section}{List of Listings}
	\section*{List of Listings}
	\addcontentsline{toc}{section}{List of Listings}
\end{document}
