\documentclass[11pt,a4paper]{article}
\usepackage[T1]{fontenc}
\usepackage[utf8]{inputenc}
\usepackage{tikz}
\usepackage{listings}
\usetikzlibrary{shadows}

% Code Sytles and colours
\usepackage{color}
 
\definecolor{codegreen}{rgb}{0,0.6,0}
\definecolor{codegray}{rgb}{0.5,0.5,0.5}
\definecolor{codepurple}{rgb}{0.58,0,0.82}
\definecolor{backcolour}{rgb}{0.95,0.95,0.92}
 
\lstdefinestyle{mystyle}{
    backgroundcolor=\color{backcolour},   
    commentstyle=\color{codegreen},
    keywordstyle=\color{magenta},
    numberstyle=\tiny\color{codegray},
    stringstyle=\color{codepurple},
    basicstyle=\footnotesize,
    breakatwhitespace=false,         
    breaklines=true,                 
    captionpos=b,                    
    keepspaces=true,                 
    numbers=left,                    
    numbersep=5pt,                  
    showspaces=false,                
    showstringspaces=false,
    showtabs=false,                  
    tabsize=2
}
 
\lstset{style=mystyle}



\newcommand*\keystroke[1]{%
  \tikz[baseline=(key.base)]
    \node[%
      draw,
      fill=white,
      drop shadow={shadow xshift=0.25ex,shadow yshift=-0.25ex,fill=black,opacity=0.75},
      rectangle,
      rounded corners=2pt,
      inner sep=1pt,
      line width=0.5pt,
      font=\scriptsize\sffamily
    ](key) {#1\strut}
  ;
}
\title{ Help $Vim$ in \LaTeX }
\author{Tongust}
\date{\today}
\renewcommand*\contentsname{Summary}
\begin{document}
	\maketitle
	\tableofcontents
	\section{Execute the shell command without exiting vim.}	%%%%%%%%%%%%%%%%%%%

	While \LaTeX ing, you could just use \keystroke{ESC} : {\color{red} !pdflatex \%} to compile the main file without exiting the vim. 

	\addcontentsline{toc}{section}{show the PDF}
	\section*{Show the pdf generated by \LaTeX }

	Type the command \keystroke{:!envince vim.pdf} to open the pdf.

	\section{Jump to  the begin or end of C-style code block}	%%%%%%%%%%%%%%%%%%%%

	To jump to the beginning of a C code block, use the \keystroke{\lbrack\{} command. To jump to the end of a C code block, use the \keystroke{\rbrack\}} command.

	\section{Split the Screen}	%%%%%%%%%%%%%%%%%%%%%%%%%%%

    Try this:
    \begin{verbatim}
    :vs %:p:h/otherfile
    as %:p:h gives you the path of the current file. 
    \end{verbatim}
    :vs \%:p:h/otherfile
    as \%:p:h gives you the path of the current file
	\keystroke{:ls  } list of open buffer.\\
	\keystroke{:bp } jump to previous buffer.\\
	\keystroke{:bn } move to the n'th buffer.\\
	\keystroke{:b tab } with tab-key providing auto-completion.\\
	\keystroke{\^} to switch to the list file you were working on.\\
	You can save sessions of vim: \\
	\emph{:mksession! ~/today.ses}\\
	Then you can load that session by using
	\emph{vim -S ~/today.ses}

\end{document}
